\documentclass[12pt]{article}
\usepackage[finnish]{babel}
\linespread{1.5}
\begin{document}

\title{TA5 Makrotaloustieteen erikoiskurssi}
\author{Oppimispäiväkirja, Tuomas Starck, 013464031}
\date{Kevät 2020}
\maketitle

%  %%  %  %%  %  %%  %  %%  %  %%  %  %%  %  %%  %  %%  %  %%  %
%
% TODO
%  Tarkista nimikkeet
%
%  %%  %  %%  %  %%  %  %%  %  %%  %  %%  %  %%  %  %%  %  %%  %
%
% Nosta esiin pääaihe tai keskeiset aiheet (tärkeä tai mielenkiintoinen)
% Kommentoi ja käytä omaa ääntä
% Esimerkiksi:
%  Mikä oli keskeinen aihe (vrt. otsikko)?
%  Mikä oli uusi asia tai paransi taloudellista lukutaitoasi?
%  Millä tavoin luento auttoi ymmärtämään talouden toimintaa paremmin?
%  Mikä jäi epäselväksi?
%  Mitä vastaväitteitä heräsi?
%
%  %%  %  %%  %  %%  %  %%  %  %%  %  %%  %  %%  %  %%  %  %%  %

\newpage
\section{Finanssikriisin syitä ja seurauksia}

Anne Mikkola aloitti orkestroimansa kurssin luennolla ''Finanssikriisin syitä ja seurauksia''. Luento käsitteli finanssikriisejä yleisesti ja erityisesti vuosina 2007–2009 maailmantaloutta ravistellutta kriisiä ja sitä, kuinka se oli syntynyt USA:n asunto- ja rahoitusmarkkinoilla. Koska tunnin alussa käsiteltiin kurssiin liittyviä asioita, luennon varsinaiselle aiheelle jäi hieman tavallista vähemmän aikaa, mutta vastapainoksi tämä luento oli sisältönsä puolesta yksi kurssin teknisemmistä.

2000-luvun alussa Yhdysvaltain asuntomarkkinoilla kehiteltiin finanssi-innovaatioita, jotka mahdollisti myös verrattain köyhien ihmisten saada asuntolainaa. Nämä eri finanssikyhäelmät – M/ABS:t, CDO:t, jne – ovat mielestäni kiehtovia, mutta ne myös auttoivat hämärtämään sitä tosiasiaa, että lainojen taustalla olevien vakuuksien arvo ei vastannut lainojen arvoa. Korkojen lasku ja lainan saannin helppous myös ruokki asuntojen hintakehitystä ja vuosien 2001–2005 välillä asuntojen mediaanihinta nousi lähes kolmanneksen.

Kuplan puhkeaminen alkoi asuntojen hintojen nousun pysähtymisellä tai laskulla, jonka seurauksena velkojen maksukyky heikkeni. Velallisen saattoi olla edullisempaa jättää asunto kuin maksaa siitä arvoaan enemmän. Seurauksena on kierre, jossa reaaliomaisuuden ja lainojen arvot laskevat, lainanannon kannattavuus laskee ja lainanantoa rajoitetaan, talous supistuu, ja nämä kaikki seuraukset ruokkivat toisiaan.

Luennolla kerrottiin kuinka pankkien epäluottamus toisiinsa paisutti Yhdysvaltojen asuntomarkkinoiden kriisin globaaliksi ongelmaksi ja että Euroopassa valtioiden budjettien alijäämät kasvoivat. Näin jälkeenpäin luentoa ja materiaaleja selatessani huomaan, että olisin mielelläni oppinut lisää finanssikriisin vaikutuksista Atlannin tällä puolen. Aihetta käsiteltiin toki aikanaan paljon uutisissa, mutta tarkempi taloustieteellinen katsaus olisi minusta kiinnostava.

Finanssikriisistä toipuminen tapahtui eri tavoilla eri puolilla maailmaa. USA:ssa rahoituslaitoksia sekä kaatui, tuettiin, ja sosialisoitiin. Lisäksi vuoden 2009 alussa yli 700 miljardin elvytyspaketti hyväksyttiin kongressissa. Myös Japanissa käytettiin satoja miljardeja elvytykseen. EKP laski ohjauskorkoa ja myönsi eurooppalaisille pankeille halpoja lainoja, jotta rahaliikenne pankkien välillä toimisi. Vaikka talouden supistuminen oli tietysti haitaksi, välttyi Suomi varsinaiselta pankkikriisiltä.

Vaikka vuoden 2008 finanssikriisi oli ennestään uutisista tuttu, oli tämä luento kokonaisuudessaan kiinnostava, sillä sain sekä tarkemman että koherentimman kuvan syistä ja seurauksista (erityisesti tapahtumista Yhdysvalloissa). Lisäksi nyt ymmärrän myös 90-luvun laman hieman paremmin.


\newpage
\section{Riittääkö Suomessa työvoima?}

Turun yliopiston professori Matti Virén luennoi otsikolla ''Riittääkö työvoima Suomessa?''. Suurten ikäluokkien siirtyessä kohti eläkettä, otsikon kysymys vaikuttaa ajankohtaiselle ja odotin kuulevani luennolla ennustuksia väes\-tö\-ra\-ken\-teen ja työvoiman tarpeen kehityksestä. Välittömästi luennon alettua Virén kuitenkin -- omien sanojensa mukaan -- banalisoi aiheen otsikon ja kiinnitti huomion kysymykseen siitä, mitä tarkoittaa työvoiman riittäminen tai työvoiman tarve.

Aiheen taustalla on toki todellinen muutos eli väestön ikääntyminen. Vuoden 2010 huipun jälkeen työikäinen väestö on tosiasiallisesti supistunut jo 100 000 hengellä, joka on huomattavasti suurempi luku kuin mitä olisin arvannut sen olevan. Virén kuitenkin painotti joustojen merkitystä tässä keskustelussa.

Joustot ovat tietysti tuttu käsite ensimmäisiltä taloustieteen luennoilta alkaen. Tässä yhteydessä luennon oleellinen viesti oli, ettei nykyisten lukujen perusteella voi sellaisenaan tehdä kovin pitkälle vietyjä ennustuksia tulevasti työvoiman tarpeesta, sillä tarve itsessään ei ole kiveen hakattu. Tai kuten Virén asian muotoili: ''Kun puhutaan työvoiman palkkajoustosta, sen käänteiskuvaus on tietenkin silloin palkkojen jousto työllisyyden suhteen.'' Toisin sanoen työvoimamarkkinat ovat talouden pelikenttä, jossa tasapaino määräytyy monien tekijöiden perusteella. Jos esimerkiksi jollain alalla työvoiman tarve kasvaa, nousee myös palkat ja sitä kautta työvoiman tarjonta. Vastaavasti jos työvoimakustannukset nousevat, muuttuu robotisaatio suhteellisesti edullisemmaksi vaihtoehdoksi.

Edellä esitetyn päättelyketjun perusteella vastaus luennon otsikon kysymykseen ei ole koskaan kyllä tai ei vaan esimerkiksi se, että työvoiman kysynnän kasvu voi johtaa alan automaatiotason ja rakenteiden muutokseen.

Kokonaisuudessaan pidin tästä luennosta erityiseksi siksi, että Virén avasi aiheen perustekijöihinsä ja selitti niistä koostuvan rakenteen. Luonnollisesti luento sukelsi yksityiskohtiin ja tilastoihin paljon tarkemmin. Mielenkiintoisia yksityiskohtia olivat mielestäni esimerkiksi ne, että työvoiman hintajousto on tyypillisesti lähellä lukua $-0,5$ ja se, millä tavoin eri työllisyyslukuja luodaan (Työ- ja elinkeinoministeriön kortistoluvut ja Tilastokeskuksen työ\-voi\-ma\-tut\-ki\-muk\-sen luvut). Piilotyöttömyys ymmärrettävästi herätti luennolla kysymyksiä ja keskustelua, sillä noin 150 000 työttömän kadottaminen kuulostaa merkittävältä, mutta jätän tilastojen käsittylen toiseen kertaan.


\newpage
\section{Johdatus eläkejärjestelmiin}

Eläketurvakeskuksen johtaja ja tohtori Jaakko Kiander luennoi otsikolla ''Johdatus eläkejärjestelmiin''. Otsikko oli hyvin osuva, sillä luento oli sisältönsä puolesta suoraviivainen katsaus eläkejärjestelmiin eikä juuri muuta. Kiander ei sisällyttänyt luentoon omaa työnantajaansa, joten lähdin etsimään tietoa itse. Opin että on olemassa laki Eläketurvakeskuksesta, joka sanoo, että ETK:n tehtäviin kuuluu eläketurvan toimeenpano ja kehittäminen, tutkimustoiminta, tiedottaminen, eläkelaitosten yhteistyön edistäminen ja ainakin 12 muuta lain luettelemaa kohtaa.

Suomessa on käytössä julkinen, lakisääteinen, ansiosidonnainen, etuusperusteinen, osittain rahastoiva työeläkejärjestelmä, jota pääsääntöisesti toteuttaa yksityiset työeläkejärjestöt. Maailmalla on myös erillaisia ratkaisuja. Esimerkiksi eläkkeiden rahoitus voisi toimia jakojärjestelmällä, jossa maksettavat eläkkeet rahoitetaan suoraan verotuloilla. Vastaavasti voidaan käyttää rahastoja siten, että eläkemenot kerätään etukäteen, sijoitetaan ja eläkkeet maksetaan varallisuudesta. Suomen malli on ei ole puhdas versio kummastakaan näistä vaan välimuoto.

Suomessa käytössä oleva etuusperusteisuus tarkoittaa sitä, että eläkettä maksetaan etukäteen määrättyjen karttumasääntöjen mukaan. Vaihtoehtoisesti maksuperusteinen järjestelmä tarkoittaa sitä, että eläkkeen suuruus määräytyy työuran aikana maksettujen maksujen ja niille kertyneiden tuottojen perusteella. Luonnollisesti suurempi ansiotaso vaikuttaa positiivisesti eläkekertymään.

Suomen järjestelmän vahvuuksia on kattavuus ja se, että se on hyvin helppo ja automaattinen palkansaajan näkökulmasta. Kiander totesi työeläkeyrityksiin liittyen, ''yleensä ihmiset eivät välttämättä edes tiedä sitä itse,'' mikä osui naulan kantaan, sillä en osaa suoralta kädeltä mainita minkään työnantajani valitsemaa työeläkeyritystä. Yleisemmin ottaen tämä kuvaa suhdettani koko luennon aiheeseen -- vaikka olen ollut työelämässä yli vuosikymmenen, en ole ajatellut eläkkeisiin liittyviä asioita vielä kertaakaan, joten tämä johdatusluento on yhtä aikaa sekä hyvin hyödyllinen että kauttaaltaan uutta tietoa.

Koska työeläkejärjestelmä hajauttaa riskit kollektiivisesti on se paitsi helppo niin myös riskitön yksittäisen työntekijän näkökulmasta ja karttuneiden etuuksien suoja on vahva. Sen sijaan yhteiskunnan muutos laajemmin aiheuttaa riskejä koko järjestelmän toimivuudelle. Syy on sama kuin taloudessa tällä hetkellä yleisesti: väestön ikääntyminen ja huoltosuhteen heikkeneminen. Haasteet eivät toki koske Suomea erityisesti vaan koko Eurooppa niin sanotusti japanisoituu. Tulevaisuus tuo mukanaan paineita korottaa työeläkemaksuja, jotta rahastot eivät ehtyisi.


\newpage
\section{Työmarkkinoiden toiminta ja rakenneuudistukset}

Jyväskylän yliopiston kauppakorkeakoulun työelämäprofessori Seija Ilmakunnas luennoi otsikolla ''Työmarkkinoiden toiminta ja rakenneuudistukset''. Luento tarkasteli suomalaisen hyvinvointiyhteiskunnan edellytyksiä erityisesti työmarkkinoiden, palkkojen ja kilpailukyvyn näkökulmasta. Lisäksi julkinen keskustelu sai ripauksen kritiikkiä osakseen, sillä keskusteluissa käytetään usein poliittisista syistä lukuja, jotka Ilmakunnaan mukaan ovat liian tarkkoja suhteessa siihen, mitä asiasta tosiasiassa voidaan tietää.

Rakenteellinen työttömyys on asia, josta on puhuttu julkisuudessa ainakin 90-luvun lamasta lähtien (ehkä pidempään, mutta oma muistini ei kanna kauemmas). Tämä luento tarjosi ensimmäistä kertaa analyyttisen haltuunoton aiheesta. Tiesin että rakenteellinen työttömyys käsitteenä pyrkii jakamaan työttömyyden kahtia ja erottamaan niistä suhdanteista riippuvan osuuden, jotta työttömyyttä voidaan paremmin käsitellä. Sen sijaan en tiennyt, kuinka kehnosti rakenteellista työttömyyttä kyetään mittaamaan tai arvioimaan. Ilmakuntaan esittämä diagrammi siitä, kuinka rakenteellisen työttömyyden arvio on seurannut työttömyyttä itsessään laman vuosina 1990--95 oli silmiä avaava.

Muita mielenkiintoisia asioita luennolla olivat työllisyyteen ja rakenteisiin liittyvät mekanismit ja muutokset viime vuosikymmeniltä. Esimerkiksi Beveridge-käyrien esittely tarjosi paremman käsityksen kohtaanto-ongelmaan ja Okunin laki -- joka ei varsinaisesti ole laki vaan empiirinen havainto -- siihen, kuinka vahvasti talouskasvu ja työllisyys kulkevat käsi kädessä.

Mutta luennon pääasia oli hyvinvointiyhteiskunta. Suomalaisessa mallissa ideana on, että talouskasvu, jota tarvitaan mallin ylläpitämiseksi, haetaan nimenomaan viennistä, sillä kotimaiset markkinat ovat pienet. Koska vienti on riippuvainen muiden markkinoiden kysynnästä, täytyy Suomen kyetä sopeutumaan nopeasti, jos kysynnässä tapahtuu muutoksia. Perinteisesti Suomessa on ollut jollain tavalla toteutunut yhteiskuntasopimus, jossa toisaalta työntekijäosapuoli osallistuu tukemaan viennin kustannuskilpailukykyä ja vastaavasti työnantajaosapuoli ottaa osaa kollektiiviseen sosiaalisen turvaverkon ylläpitoon. Mielenkiintoinen huomio, jota en ole tullut aiemmin ajatelleeksi, oli se, että sosiaaliturva tukee talouden muutosvalmiutta, ketteryyttä ja sitä kautta mahdollistaa teknologien kehityksen.

Olin pari vuotta sitten HSL:llä töissä samaan aikaan, kun kikysopimus oli tullut voimaan. Silloin ylimääräinen 6 minuuttia työpäivässä ei vaikuttanut millään tavalla mielekkäältä. Työntekijän näkökulmasta niin pieni muutos työajassa ei voinut olla hyödyllinen kenellekään -- lisäsi ehkä vaan byrokratiaa. En varsinaisesti osaa tämän luennon perusteella vieläkään sanoa, oliko juuri niillä minuuteilla kansantaloudelle merkitystä, mutta näin jälkikäteen on myönnettävä, että Sipilän hallitus onnistui nostamaan työllisyysastetta ja sillä on ainakin merkitystä niille ihmisille, jotka saivat töitä.


\newpage
\section{Rahapolitiikka}

Valtiotieteen tohtori Antti Suvanto luennoi aiheella ''Rahapolitiikka''. Luento alkoi ymmärrettävästi johdannolla, joka kattoi peruskäsitteistön määrittelyn ensimmäisenä kysymyksenään: ''Mitä on raha?'' Vaikka kysymys saattaa vaikuttaa banaalille, arvostan itse suuresti luennon lähestymistapaa lähteä liikkeelle perusteista -- tosiasiassa harvoin sitä tulee arjessa ajateltua rahan luonnetta. Pitkälti tämän kurssin ansiosta olen alkanut etsiä kirjoja, jotka käsittelevät rahan filosofiaa. Erityisesti tahtoisin ymmärtää paremmin kuinka luotonlaajennus ja vähimmäisvarantojärjestelmä toimii ja mitä implikaatioita niillä on.

Rahapolitiikka on keskuspankin harjoittamaa talouspolitiikkaa. Suomi on osa euroaluetta, joten rahapolitiikan toiminnallinen keskus on vuodesta 1999 alkaen ollut Euroopan keskuspankki. Rahapolitiikan ensisijainen tavoite on hintavakaus eli inflaation pitäminen tavoitetasolla. Liian korkea inflaatio johtaisi sattumanvaraiseen varallisuuden jakoon ja olisi lisäriski, sillä talouden ennustettavuus heikkenee. Deflaatio sen sijaan aiheuttaisi investointien tyrehtymistä, joten EKP-alueella tavoitteena on pitää inflaatio lähellä muttei yli kahden prosentin. Hintavakauden lisäksi toissijaisia rahapolitiikan tavoitteita ovat talouskasvun ja työllisyyden tukeminen. Keskuspankin käytettävissä olevia keinoja ovat ohjauskorko, arvopaperiostot ja viestintä.

Mielestäni oli suorastaan huvittavaa kuulla, kun Suvanto kertoi, että ennen 80- ja 90-lukuja rahapolitiikkaa luotiin salassa, päätöksillä pyrittiin yllättämään markkinat ja tätä pidettiin hyvänä käytäntönä. Olen seurannut talouden tapahtumia ilmeisen lyhyen aikaa, sillä avoimuus ja ennen kaikkea ennakoitavuus tuntuvat itsestäänselvyyksille. Tutummalta kuulosti Suvannon kuvaus siitä, kuinka EKP ohjauskoron muutoksella ei ole tarkoitus aiheuttaa muutosta markkinoilla, sillä huolella hoidettu viestintä ennen päätöksentekoa on saanut markkinat odottamaan päätöstä ja sopeutuminen on jo tapahtunut.

Luennon lopulla oli vielä katsaus finanssikriisiin ja sen syntytekijöihin. 2000-luvun alussa rahapolitiikka onnistui päätavoitteessaan ja inflaatio oli toivotulla tasolla, mutta rahoitusmarkkinoiden vakauden tai euroalueen sisäisen vaihtotaseen eroja ei huomioitu riittävästi. Puoli vuotta sen jälkeen, kun Yhdysvaltojen subprime-luotot nousivat huolenaiheeksi, levisi ongelma myös euroalueelle ja pankkien välinen rahakauppa tyrehtyi. EKP reagoi tähän antamalla 100 miljardin euron luottopaketin rahaliikenteen elvyttämiseksi. Talouden näkymät synkkenivät merkittävät vuoden 2008 loppupuolella ja kaikki keskuspankit laskivat ohjauskorkoaan synkronoidusti. Finanssikriisi nosti pintaan eurooppalaisen velkakriisin, josta seurasi myöhemmin koko euroalueen luottamuskriisi. Lopulta kurinalaisen talouspolitiikan avulla näistä kaikista selvittiin vaihtelevan onnistuneesti vain talouden supistumisen kustannuksella. Mikään valtio ei luopunut eurosta.


\newpage
\section{Asuntomarkkinat ja makrovakaus}

Valtion taloudellisen tutkimuskeskuksen tutkimusjohtaja Essi Eerola luennoi otsikolla ''Asuntomarkkinat ja makrovakaus''. Luennon teemoja olivat Suomen asuntomarkkinoiden kehitys, lainanotto ja makrovakaus. Koska kaikki asuntomarkkinoihin liittyvä on itselleni nyt ajankohtaista, olin erityisen kiinnostunut tämän luennon sisällöstä.

Yllätyin kun tutkimusjohtaja Eerola esitteli asuntojen reaalihintaindeksin kehitystä viimeiseltä 30 vuodelta, jossa näkyi, kuinka pääkaupunkiseudun hintataso alkoi eriytyä muun Suomen hintatasosta kymmenisen vuotta sitten. Yllätyin siksi, että luulin kaupungistuminen Suomessa olleen vahvaa jo ainakin 90-luvun lopulta alkaen ja olisin olettanut sen näkyneen asuntojen hintakehityksessä vähiintään niiltä ajoilta alkaen. Toisaalta sama reaalihintaindeksi paljastaa myös sen, että asuntojen hintakehitys koko 2000-luvun alun on suurelta osin ollut vain korjausliikettä 90-luvun laman jyrkälle pudotukselle, jolloin koko maassa asuntojen hinnoista katosi keskimäärin noin puolet. Tätä taustaa vasten kuvaajan lukuja oli helpompi ymmärtää.

Mielestäni tärkein aihe luennolla olivat lainanotto ja makrovakaus. Lainanoton mahdollisuus itsessään on hyödyllistä, sillä se lisää hyvinvointia ja antaa esimerkiksi nuorille mahdollisuuden velkaantua opintoja varten ja maksaa opintovelkaa pois myöhemmin työelämässä. Tai kuten taloustieteessä asia ilmaistaan: tasata kulutusta yli elinkaaren. Mutta liiallinen velkaantuminen on tietysti myös riski paitsi yksilölle niin myös laajemmin talouden näkökulmasta. Jos talous heikkenee saattaa paljon asuntoja tulee myyntiin yhtä aikaa, sillä ihmiset eivät välttämättä selviä velkataakastaan. Tämä entisestään laskee asuntojen arvoa, vaikeuttaa lainanottajien asemaa, supistaa kulutusta ja ruokkii talouden heikkenemistä.

Lainanottoa pyritään rajoittamaan sääntelyllä. Lainan suuruutta saatetaan rajoittaa suhteessa hankittavan asunnon arvoon nähden tai vaihtoehtoisesti lainanottajan tuloihin nähden. Erityisesti finanssikriisin jälkeen on otettu käyttöön makrovakaustoimia ja lainoituksen sääntelyä on kiristetty, mutta samalla Eerola huomautti, että tosiasiassa sääntelyn tehokkuus ja vaikutukset eivät ole kovin kattavasti tunnettuja. Joka tapauksessa esimerkiksi Suomessa on käytössä enimmäisluototussuhde (85\% tai 95\% ensiasunnonostajille), joka rajaa kotitalouksien lainataakkaa suhteessa asunnon arvoon.

% TODO Jokin lopetus


\newpage
\section{Rahoitusmarkkinoiden sääntely ja valvonta}

Finanssivalvonnan pankkivalvonnan osastopäällikkö Jyri Helenius luennoi otsikolla ''Rahoitusmarkkinoiden -- erityisesti pankkien -- sääntely ja valvonta''. Finanssivalvonta edistää ''finanssimarkkinoiden vakautta ja luottamusta sekä asiakkaiden, sijoittajien ja vakuutettujen suojaa.'' Finanssivalvontaa säätää laki finanssivalvonnasta. Helenius painotti valvomisen roolia sääntelyn sijaan, mutta Finanssivalvonta osallistuu finanssimarkkinoiden lainsäädännön valmisteluun, vaikkei päättävä hallinnon elin olekaan. Valvottavia kohteita on pitkä lista, joka kattaa mm. pankit, vakuutusyhtiöt ja -kassat, työttömyyskassat, sijoituspalveluyritykset ja pörssin.

Nyt kun finanssikriisi on vielä tuoreessa muistissa, ei väitettä valvonnan ja turvaverkkojen tarpeellisuutta tarvitse perustella kenellekään. Sen sijaan on täysin perusteltua muistuttaa, ettei turvallisuutta saa ilmaiseksi, sillä sääntelyllä voi olla sivuvaikutuksia ja toisaalta turvaverkko voi kannustaa pankkeja ottamaan suurempia riskejä.

Erityisesti ''Rahapolitiikka''-luennon jälkeen oli mielenkiintoista kuulla, mitä finanssikriisistä tarkalleen ottaen opittiin. Helenius listasi neljä merkittävää kohtaa. Ensimmäinen oli arvattavasti se, että pankkien riskimarginaalit olivat liian ohuet. Kriisin jälkeen sääntelyä on kiristetty näiltä osin nostamalla pääoma- ja likviditeettivaatimuksia sekä vä\-him\-mäis\-oma\-va\-rai\-suus\-as\-tet\-ta. Toinen ongelma liittyi sääntelyyn itseensä, sillä sääntelyn vaatimukset tiukentuivat kriisin edetessä ja sitä kautta hankaloittivat tilanteen toipumista. Nyttemmin sääntelyn pääomavaatimuksia on muokattu vastasyklisemmiksi.

Kolmas ongelma liittyi johdon kannustimiin, jotka olivat mitoitettu palkitsemaan riskinottoa liian lyhyellä aikavälillä. Kriisin seurauksena bonuksien maksujen ehtoja ja viiveitä on pidennetty siten, ettei vakavaraisuuden vaarantamisesta palkita niin herkästi. Viimeisenä ongelmana Helenius mainitsi systeemiriskit, joita on kriisin jälkeen alettu torjua uusilla makroriskeihin perustuvilla vakavaraisuusvaatimuksilla ja tehostetulla valvonnalla. Lisäksi pankkien kriisinratkaisumallia on muutettu sellaiseksi, ettei pankkeja tueta verovaroilla vaan pikemminkin sijoittajien varoilla.

Finanssivalvonta ei ole pelkkä paperitiikeri, sillä jos valvottava taho ei toimi sääntöjen mukaisesti, voi Finanssivalvonta tarttua sanktioihin. Ääripäissä vaihtoehtoja ovat sakkorangaistus ja koko toimiluvan poisotto. Keskitien vaihtoehto on se, että Finanssivalvonta voi asettaa asiamiehen tarkkailemaan valvottavaa organisaatiota. Se oli minusta mielenkiintoista.


\newpage
\section{Hitaan kasvun anatomia finanssikriisin jälkeen}

Suomen Pankin johtokunnan jäsen professori Seppo Honkapohja luennoi otsikolla ''Hitaan kasvun anatomia finanssikriisin jälkeen''. Suomi on saanut nauttia talouden ja elintason kasvusta lähes koko olemassaolonsa ajan jos sodat ja 90-luvun lama pyöristetään pois tarkastelusta. Kuitenkin finanssikriisin jälkeisinä vuosina 2008--2014 BKT ei kasvanut ollenkaan ja senkin jälkeen talous on voinut aneemisesti. Honkapohja käsitteli luennollaan syitä sille, mistä talouden heikko kehitys Suomessa on johtunut.

Finanssikriisistä lähtien sekä korot että inflaatio ovat olleet ja pysyneet matalana. Vaikka tilanne on uusi, se ei riitä selittämään Suomen heikkoa talouskehitystä, sillä sekä USA että muut Euroopan maat ovat tilanteesta huolimatta toipuneet finanssikriisistä vähiintään kohtuullisesti. Yksi ilmeinen Suomelle erityinen tekijä on Nokia ja sen matkapuhelintuotannon katoaminen. Vaikka Nokia oli merkittävä Suomelle, se ei Honkapohjan mukaan kuitenkaan riitä selittämään kuin osan taloustilanteesta.

Kuten Ilmakunnaan luennolla aiemmin opimme, Suomen hyvinvointimalli on riippuvainen viennin kilpailukyvystä. Vienti heikkeni voimakkaasti juuri finanssikriisin jälkeisinä aikoina. Osittain se johtui globaalista talouden kurjistumisesta, mutta Suomen ongelmana oli myös kasvaneet yksikkötyökustannukset, jotka söivät kustannuskilpailukykyä. Kustannusten kasvu selittyy ainakin osin sillä, että sopivasti ennen Yhdysvaltain asuntokuplan puhkeamista Suomessa työntekijäosapuoli oli onnistunut sopimaan itselleen reilut palkankorotukset, jotka valitettavasti olivat kilpailukyvyn riippakivenä monta seuraavaa vuotta.

Heikko vienti, Nokian menetys ja yleisesti vaikea ja epävarma tilanne maailmantaloudessa on varsin kattava ongelmavyyhti. Lisäksi vielä investoinnit romahtivat ja Suomi ei enää ollut PISA-tulosten kärjessä niin kuin pitkään aiemmin. Honkapohja esitti luennon lopuksi suosituksia siitä, mitä tilanteen korjaamiseksi tulisi tehdä. Kuten edellisestä voi päätellä, listalla on muun muassa investointeihin ja kehitykseen panostaminen, koulutustason nosto ja kustannuskilpailukyvyn parantaminen.

Tulevaisuuden haasteiksi Honkapohja mainitsi väestön ikääntymisen ja ilmastonmuutoksen. Saatan itse olla liiankin optimisti, mutta itse pidän ilmastonmuutosta mahdollisuutena, sillä sen vaatimat muutokset ovat myös mahdollisuus talouskasvun näkökulmasta.


\newpage
\section{Tuottavuus, talouskasvu ja yhteiskunta}


\newpage
\section{Keskustelua globalisaatiosta}

\end{document}
