\documentclass[12pt]{article}
\usepackage[finnish]{babel}
\linespread{1.5}
\begin{document}

\title{TA5 Makrotaloustieteen erikoiskurssi}
\author{Oppimispäiväkirja, Tuomas Starck, 013464031}
\date{Kevät 2020}
\maketitle

%  %%  %  %%  %  %%  %  %%  %  %%  %  %%  %  %%  %  %%  %  %%  %
%
% Nosta esiin pääaihe tai keskeiset aiheet (tärkeä tai mielenkiintoinen)
% Kommentoi ja käytä omaa ääntä
% Esimerkiksi:
%  Mikä oli keskeinen aihe (vrt. otsikko)?
%  Mikä oli uusi asia tai paransi taloudellista lukutaitoasi?
%  Millä tavoin luento auttoi ymmärtämään talouden toimintaa paremmin?
%  Mikä jäi epäselväksi?
%  Mitä vastaväitteitä heräsi?
%
%  %%  %  %%  %  %%  %  %%  %  %%  %  %%  %  %%  %  %%  %  %%  %

\newpage
\section{Finanssikriisin syitä ja seurauksia}

% TODO tarkista nimike
Tohtori Anne Mikkola aloitti orkestroimansa kurssin luennolla ''Finanssikriisin syitä ja seurauksia''. Luento käsitteli finanssikriisejä yleisesti ja erityisesti vuosina 2007–2009 maailmantaloutta ravistellutta kriisiä ja sitä, kuinka se oli syntynyt USA:n asunto- ja rahoitusmarkkinoilla. Koska tunnin alussa käsiteltiin kurssiin liittyviä asioita, luennon varsinaiselle aiheelle jäi hieman tavallista vähemmän aikaa, mutta vastapainoksi tämä luento oli sisältönsä puolesta yksi kurssin teknisemmistä.

2000-luvun alussa Yhdysvaltain asuntomarkkinoilla kehiteltiin finanssi-innovaatioita, jotka mahdollisti myös verrattain köyhien ihmisten saada asuntolainaa. Nämä eri finanssikyhäelmät – M/ABS:t, CDO:t, jne – ovat mielestäni kiehtovia, mutta ne myös auttoivat hämärtämään sitä tosiasiaa, että lainojen taustalla olevien vakuuksien arvo ei vastannut lainojen arvoa. Korkojen lasku ja lainan saannin helppous myös ruokki asuntojen hintakehitystä ja vuosien 2001–2005 välillä asuntojen mediaanihinta nousi lähes kolmanneksen.

Kuplan puhkeaminen alkoi asuntojen hintojen nousun pysähtymisellä tai laskulla, jonka seurauksena velkojen maksukyky heikkeni. Velallisen saattoi olla edullisempaa jättää asunto kuin maksaa siitä arvoaan enemmän. Seurauksena on kierre, jossa reaaliomaisuuden ja lainojen arvot laskevat, lainanannon kannattavuus laskee ja lainanantoa rajoitetaan, talous supistuu, ja nämä kaikki seuraukset ruokkivat toisiaan.

Luennolla kerrottiin kuinka pankkien epäluottamus toisiinsa paisutti Yhdysvaltojen asuntomarkkinoiden kriisin globaaliksi ongelmaksi ja että Euroopassa valtioiden budjettien alijäämät kasvoivat. Näin jälkeenpäin luentoa ja materiaaleja selatessani huomaan, että olisin mielelläni oppinut lisää finanssikriisin vaikutuksista Atlannin tällä puolen. Aihetta käsiteltiin toki aikanaan paljon uutisissa, mutta tarkempi taloustieteellinen katsaus olisi minusta kiinnostava.

Finanssikriisistä toipuminen tapahtui eri tavoilla eri puolilla maailmaa. USA:ssa rahoituslaitoksia sekä kaatui, tuettiin, ja sosialisoitiin. Lisäksi vuoden 2009 alussa yli 700 miljardin elvytyspaketti hyväksyttiin kongressissa. Myös Japanissa käytettiin satoja miljardeja elvytykseen. EKP laski ohjauskorkoa ja myönsi eurooppalaisille pankeille halpoja lainoja, jotta rahaliikenne pankkien välillä toimisi. Vaikka talouden supistuminen oli tietysti haitaksi, välttyi Suomi varsinaiselta pankkikriisiltä.

Vaikka vuoden 2008 finanssikriisi oli ennestään uutisista tuttu, oli tämä luento kokonaisuudessaan kiinnostava, sillä sain sekä tarkemman että koherentimman kuvan syistä ja seurauksista (erityisesti tapahtumista Yhdysvalloissa). Lisäksi nyt ymmärrän myös 90-luvun laman hieman paremmin.


\newpage
\section{Riittääkö Suomessa työvoima?}

% TODO tarkista nimike
Turun yliopiston professori Matti Virén luennoi otsikolla ''Riittääkö Suomessa työvoima?''.

\end{document}
