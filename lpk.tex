\documentclass[12pt]{article}
\usepackage[finnish]{babel}
\linespread{1.5}
\begin{document}

\title{TA5 Makrotaloustieteen erikoiskurssi}
\author{Oppimispäiväkirja, Tuomas Starck, 013464031}
\date{Kevät 2020}
\maketitle

%  %%  %  %%  %  %%  %  %%  %  %%  %  %%  %  %%  %  %%  %  %%  %
%
% TODO
%  Tarkista nimikkeet
%
%  %%  %  %%  %  %%  %  %%  %  %%  %  %%  %  %%  %  %%  %  %%  %
%
% Nosta esiin pääaihe tai keskeiset aiheet (tärkeä tai mielenkiintoinen)
% Kommentoi ja käytä omaa ääntä
% Esimerkiksi:
%  Mikä oli keskeinen aihe (vrt. otsikko)?
%  Mikä oli uusi asia tai paransi taloudellista lukutaitoasi?
%  Millä tavoin luento auttoi ymmärtämään talouden toimintaa paremmin?
%  Mikä jäi epäselväksi?
%  Mitä vastaväitteitä heräsi?
%
%  %%  %  %%  %  %%  %  %%  %  %%  %  %%  %  %%  %  %%  %  %%  %

\newpage
\section{Finanssikriisin syitä ja seurauksia}

Anne Mikkola aloitti orkestroimansa kurssin luennolla ''Finanssikriisin syitä ja seurauksia''. Luento käsitteli finanssikriisejä yleisesti ja erityisesti vuosina 2007–2009 maailmantaloutta ravistellutta kriisiä ja sitä, kuinka se oli syntynyt USA:n asunto- ja rahoitusmarkkinoilla. Koska tunnin alussa käsiteltiin kurssiin liittyviä asioita, luennon varsinaiselle aiheelle jäi hieman tavallista vähemmän aikaa, mutta vastapainoksi tämä luento oli sisältönsä puolesta yksi kurssin teknisemmistä.

2000-luvun alussa Yhdysvaltain asuntomarkkinoilla kehiteltiin finanssi-innovaatioita, jotka mahdollisti myös verrattain köyhien ihmisten saada asuntolainaa. Nämä eri finanssikyhäelmät – M/ABS:t, CDO:t, jne – ovat mielestäni kiehtovia, mutta ne myös auttoivat hämärtämään sitä tosiasiaa, että lainojen taustalla olevien vakuuksien arvo ei vastannut lainojen arvoa. Korkojen lasku ja lainan saannin helppous myös ruokki asuntojen hintakehitystä ja vuosien 2001–2005 välillä asuntojen mediaanihinta nousi lähes kolmanneksen.

Kuplan puhkeaminen alkoi asuntojen hintojen nousun pysähtymisellä tai laskulla, jonka seurauksena velkojen maksukyky heikkeni. Velallisen saattoi olla edullisempaa jättää asunto kuin maksaa siitä arvoaan enemmän. Seurauksena on kierre, jossa reaaliomaisuuden ja lainojen arvot laskevat, lainanannon kannattavuus laskee ja lainanantoa rajoitetaan, talous supistuu, ja nämä kaikki seuraukset ruokkivat toisiaan.

Luennolla kerrottiin kuinka pankkien epäluottamus toisiinsa paisutti Yhdysvaltojen asuntomarkkinoiden kriisin globaaliksi ongelmaksi ja että Euroopassa valtioiden budjettien alijäämät kasvoivat. Näin jälkeenpäin luentoa ja materiaaleja selatessani huomaan, että olisin mielelläni oppinut lisää finanssikriisin vaikutuksista Atlannin tällä puolen. Aihetta käsiteltiin toki aikanaan paljon uutisissa, mutta tarkempi taloustieteellinen katsaus olisi minusta kiinnostava.

Finanssikriisistä toipuminen tapahtui eri tavoilla eri puolilla maailmaa. USA:ssa rahoituslaitoksia sekä kaatui, tuettiin, ja sosialisoitiin. Lisäksi vuoden 2009 alussa yli 700 miljardin elvytyspaketti hyväksyttiin kongressissa. Myös Japanissa käytettiin satoja miljardeja elvytykseen. EKP laski ohjauskorkoa ja myönsi eurooppalaisille pankeille halpoja lainoja, jotta rahaliikenne pankkien välillä toimisi. Vaikka talouden supistuminen oli tietysti haitaksi, välttyi Suomi varsinaiselta pankkikriisiltä.

Vaikka vuoden 2008 finanssikriisi oli ennestään uutisista tuttu, oli tämä luento kokonaisuudessaan kiinnostava, sillä sain sekä tarkemman että koherentimman kuvan syistä ja seurauksista (erityisesti tapahtumista Yhdysvalloissa). Lisäksi nyt ymmärrän myös 90-luvun laman hieman paremmin.


\newpage
\section{Riittääkö Suomessa työvoima?}

Turun yliopiston professori Matti Virén luennoi otsikolla ''Riittääkö työvoima Suomessa?''. Suurten ikäluokkien siirtyessä kohti eläkettä, otsikon kysymys vaikuttaa ajankohtaiselle ja odotin kuulevani luennolla ennustuksia väes\-tö\-ra\-ken\-teen ja työvoiman tarpeen kehityksestä. Välittömästi luennon alettua Virén kuitenkin -- omien sanojensa mukaan -- banalisoi aiheen otsikon ja kiinnitti huomion kysymykseen siitä, mitä tarkoittaa työvoiman riittäminen tai työvoiman tarve.

Aiheen taustalla on toki todellinen muutos eli väestön ikääntyminen. Vuoden 2010 huipun jälkeen työikäinen väestö on tosiasiallisesti supistunut jo 100 000 hengellä, joka on huomattavasti suurempi luku kuin mitä olisin arvannut sen olevan. Virén kuitenkin painotti joustojen merkitystä tässä keskustelussa.

Joustot ovat tietysti tuttu käsite ensimmäisiltä taloustieteen luennoilta alkaen. Tässä yhteydessä luennon oleellinen viesti oli, ettei nykyisten lukujen perusteella voi sellaisenaan tehdä kovin pitkälle vietyjä ennustuksia tulevasti työvoiman tarpeesta, sillä tarve itsessään ei ole kiveen hakattu. Tai kuten Virén asian muotoili: ''Kun puhutaan työvoiman palkkajoustosta, sen käänteiskuvaus on tietenkin silloin palkkojen jousto työllisyyden suhteen.'' Toisin sanoen työvoimamarkkinat ovat talouden pelikenttä, jossa tasapaino määräytyy monien tekijöiden perusteella. Jos esimerkiksi jollain alalla työvoiman tarve kasvaa, nousee myös palkat ja sitä kautta työvoiman tarjonta. Vastaavasti jos työvoimakustannukset nousevat, muuttuu robotisaatio suhteellisesti edullisemmaksi vaihtoehdoksi.

Edellä esitetyn päättelyketjun perusteella vastaus luennon otsikon kysymykseen ei ole koskaan kyllä tai ei vaan esimerkiksi se, että työvoiman kysynnän kasvu voi johtaa alan automaatiotason ja rakenteiden muutokseen.

Kokonaisuudessaan pidin tästä luennosta erityiseksi siksi, että Virén avasi aiheen perustekijöihinsä ja selitti niistä koostuvan rakenteen. Luonnollisesti luento sukelsi yksityiskohtiin ja tilastoihin paljon tarkemmin. Mielenkiintoisia yksityiskohtia olivat mielestäni esimerkiksi ne, että työvoiman hintajousto on tyypillisesti lähellä lukua $-0,5$ ja se, millä tavoin eri työllisyyslukuja luodaan (Työ- ja elinkeinoministeriön kortistoluvut ja Tilastokeskuksen työ\-voi\-ma\-tut\-ki\-muk\-sen luvut). Piilotyöttömyys ymmärrettävästi herätti luennolla kysymyksiä ja keskustelua, sillä noin 150 000 työttömän kadottaminen kuulostaa merkittävältä, mutta jätän tilastojen käsittylen toiseen kertaan.


\newpage
\section{Johdatus eläkejärjestelmiin}

Eläketurvakeskuksen johtaja ja tohtori Jaakko Kiander luennoi otsikolla ''Johdatus eläkejärjestelmiin''. Otsikko oli hyvin osuva, sillä luento oli sisältönsä puolesta suoraviivainen katsaus eläkejärjestelmiin eikä juuri muuta. Kiander ei sisällyttänyt luentoon omaa työnantajaansa, joten lähdin etsimään tietoa itse. Opin että on olemassa laki Eläketurvakeskuksesta, joka sanoo, että ETK:n tehtäviin kuuluu eläketurvan toimeenpano ja kehittäminen, tutkimustoiminta, tiedottaminen, eläkelaitosten yhteistyön edistäminen ja ainakin 12 muuta lain luettelemaa kohtaa.

Suomessa on käytössä julkinen, lakisääteinen, ansiosidonnainen, etuusperusteinen, osittain rahastoiva työeläkejärjestelmä, jota pääsääntöisesti toteuttaa yksityiset työeläkejärjestöt. Maailmalla on myös erillaisia ratkaisuja. Esimerkiksi eläkkeiden rahoitus voisi toimia jakojärjestelmällä, jossa maksettavat eläkkeet rahoitetaan suoraan verotuloilla. Vastaavasti voidaan käyttää rahastoja siten, että eläkemenot kerätään etukäteen, sijoitetaan ja eläkkeet maksetaan varallisuudesta. Suomen malli on ei ole puhdas versio kummastakaan näistä vaan välimuoto.

Suomessa käytössä oleva etuusperusteisuus tarkoittaa sitä, että eläkettä maksetaan etukäteen määrättyjen karttumasääntöjen mukaan. Vaihtoehtoisesti maksuperusteinen järjestelmä tarkoittaa sitä, että eläkkeen suuruus määräytyy työuran aikana maksettujen maksujen ja niille kertyneiden tuottojen perusteella. Luonnollisesti suurempi ansiotaso vaikuttaa positiivisesti eläkekertymään.

Suomen järjestelmän vahvuuksia on kattavuus ja se, että se on hyvin helppo ja automaattinen palkansaajan näkökulmasta. Kiander totesi työeläkeyrityksiin liittyen, ''yleensä ihmiset eivät välttämättä edes tiedä sitä itse,'' mikä osui naulan kantaan, sillä en osaa suoralta kädeltä mainita minkään työnantajani valitsemaa työeläkeyritystä. Yleisemmin ottaen tämä kuvaa suhdettani koko luennon aiheeseen -- vaikka olen ollut työelämässä yli vuosikymmenen, en ole ajatellut eläkkeisiin liittyviä asioita vielä kertaakaan, joten tämä johdatusluento on yhtä aikaa sekä hyvin hyödyllinen että kauttaaltaan uutta tietoa.

Koska työeläkejärjestelmä hajauttaa riskit kollektiivisesti on se paitsi helppo niin myös riskitön yksittäisen työntekijän näkökulmasta ja karttuneiden etuuksien suoja on vahva. Sen sijaan yhteiskunnan muutos laajemmin aiheuttaa riskejä koko järjestelmän toimivuudelle. Syy on sama kuin taloudessa tällä hetkellä yleisesti: väestön ikääntyminen ja huoltosuhteen heikkeneminen. Haasteet eivät toki koske Suomea erityisesti vaan koko Eurooppa niin sanotusti japanisoituu. Tulevaisuus tuo mukanaan paineita korottaa työeläkemaksuja, jotta rahastot eivät ehtyisi.

\end{document}
